\documentclass[12pt,a4paper]{article}


\usepackage[left=2.5cm, right=2.5cm, top=3cm, bottom=3cm]{geometry}
\usepackage[spanish]{babel}
\usepackage{amsmath, amsthm, amssymb}
\usepackage{hyperref}
\usepackage{graphicx}
\usepackage{url}


\begin{document}

\begin{titlepage}
	\begin{center}
		\Huge
		\textbf{H.U.L.K. Interpreter}
		\vspace*{1cm}
		
		\includegraphics[width=0.75\textwidth]{HULK.jpg}
		\vspace{1cm}
		
		\LARGE
		\text{Luis Daniel Silva Martínez}
		
		
		\includegraphics[width=0.4\textwidth]{logo matcom.jpg}
		
		\Large
		School of Math and Computer Science\\
		Havana University
		
	\end{center}
\end{titlepage}

\newpage
\tableofcontents
\newpage


\section{Introducción}\label{sec:intro}
Este proyecto es un intérprete que evalúa instrucciones del lenguaje H.U.L.K. una por una, cada una terminada en punto y coma, las cuales pueden ser expresiones aritméticas, declaraciones de variables, condicionales y funciones definidas por el usuario.
\subsection{H.U.L.K.}\label{sub:hulk}
H.U.L.K. (Havana University Language of Kompilers) es un lenguaje de programación imperativo, funcional, estática y fuertemente tipado. Casi todas las instrucciones en H.U.L.K. son expresiones.
\subsection{Intérprete}\label{sub:interpreter}
Este proyecto de programación es un interpréte del lenguaje H.U.L.K. en C\texttt{\#} usando tecnología .NET Core 7.0. Tiene una solución que contiene dos proyectos:
\begin{itemize}
	\item Una biblioteca de clases (HULK\_Interpreter) donde se implementa toda la lógica de parsing y evaluación del lenguaje H.U.L.K. haciendo uso solamente la biblioteca estándar de .NET Core.
	\item Una aplicación de consola (HULK\_ConsoleInterpreter) donde se implementa la parte interactiva del intérprete.
\end{itemize}
\subsection{¿Cómo usarlo?}\label{sub:use}
	El uso del intérprete es simple, solo escribir expresiones válidas del lenguaje H.U.L.K. y presionar \texttt{ENTER}, una única instrucción a la vez, y recibirá el resultado de evaluar la expresión. En caso de algún error en las instrucciones enviadas se le notificará sobre el mismo.
\subsection{¿Cómo ejecutarlo?}\label{sub:execute}
\subsubsection*{Linux} Se debe ubicar en la carpeta principal del proyecto y ejecutar desde la terminal de Linux el comando:\\
\texttt{\bf make dev}
\subsubsection*{Windows} Ejecutar en consola desde la carpeta principal del proyecto el comando:\\
\texttt{\bf dotnet run --project HULK\_ConsoleInterface}\\
O abrir la carpeta del HULK\_ConsoleInterface y ejecutar:\\ \texttt{\bf dotnet run} 



\section{Implementación}\label{sub:impl}
Este es un ejemplo de una posible interacción:
	\begin{verbatim}
		>let x = 42 in print(x);
		42
		>function fib(n) => if (n > 1) fib(n-1) + fib(n-2) else 1;
		Function 'fib' declared succesfully
		>fib(5);
		13
		>let x = 3 in fib(x+1);
		8
		>print(fib(6));
		21
	\end{verbatim}
Cada línea que comienza con \texttt{>} representa una entrada del usuario, e inmediatamente después se imprime el resultado de evaluar esa expresión.
\subsection{Flujo}\label{sub:flow}
El proyecto sigue un camino marcado:
\begin{enumerate}
	\item Instanciación de la clase Interpreter e inicialización de la memoria con las funciones predefinidas.
	\item Recepción del código introducido por el usuario
	\item Tokenización del código y validación léxica con la clase Lexer.
	\item Parseo del código, construcción del árbol de sintaxis abstracta y validación de la sintaxis en la clase Parser.
	\item Evaluación del árbol de sintaxis abstracta y comprobación semántica en la clase Evaluator.
	\item Obtención e impresión del resultado.
	\item Nuevo código.
\end{enumerate}

\subsection{Class Interpreter}
La clase \texttt{Interpreter} es la clase principal que maneja todo el proceso del intérprete. Se encarga de coordinar el análisis léxico, el análisis sintáctico y la evaluación del código fuente. Actúa como una interfaz entre estas diferentes etapas y garantiza que el proceso de interpretación se realice de manera ordenada y correcta.
\subsection{Class Memory}
La clase \texttt{Memory} simula la memoria del intérprete. Guarda las funciones predefinidas del lenguaje y las definidas por el usuario durante la ejecución del programa.
\subsection{Class Token}
La clase \texttt{Token} representa un token individual en el análisis léxico, con su tipo de token definido en el Enum \texttt{TokenType}, su lexema y su valor en caso de tenerlo.
\subsection{Class Lexer}
La clase \texttt{Lexer} es responsable de realizar el análisis léxico. Escanea el código fuente y lo convierte en una secuencia de tokens significativos que posteriormente serán utilizados en el proceso de análisis sintáctico.
\subsection{Class Parser}
La clase \texttt{Parser} analiza la estructura sintáctica del código fuente, utilizando los tokens generados por el \texttt{Lexer}, y construye una representación estructurada del programa, un árbol de sintaxis abstracta (AST). Utiliza parsing recursivo descendente, concepto basado en la idea de dividir el problema en subproblemas más pequeños y resolverlos de manera recursiva.
\subsection{Class Evaluator}
La clase \texttt{Evaluator} es responsable de llevar a cabo la interpretación del código fuente y producir los resultados deseados según las reglas y semántica de nuestro lenguaje.
\subsection{Class Expression}
La clase \texttt{Expression} es la clase base de todas las expresiones de nuestro lenguaje H.U.L.K., de la cual heredan \texttt{BinaryExpression}, \texttt{UnaryExpression}, \texttt{LiteralExpression}, y el resto de expresiones, cada una con sus características propias.
\subsection{Class Error}
La clase \texttt{Error} hereda de Exception y permite representar y manejar errores léxicos, sintácticos y semánticos durante la ejecución del intérprete.



\end{document}